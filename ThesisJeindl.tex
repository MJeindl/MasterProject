\documentclass[10pt,a4paper]{article}
\usepackage[utf8]{inputenc}
\usepackage{amsmath}
\usepackage{amsfonts}
\usepackage{amssymb}
\usepackage{graphicx}
\usepackage[left=2cm,right=2cm,top=2cm,bottom=2cm]{geometry}
\begin{document}
\section{Experimental Setup}
Part of the experimental objective was to extend the range of the transient absorption microscope (TAM) into the UV range via SHG of the probe NOPA output. With the extension a wavelength range of 250 nm to 940 nm, with a gap between 470 nm and 500 nm, is available for probing the sample.\newline
To achieve this range a few optics, most notably the achromatic objectives used to image the sample, had to be exchanged for optics non absorbing in the UV.
\subsection{Cage setup}
To accomodate the lower wavelengths the objectives have to be exchanged for wavelengths below 350 nm, where absorption reduces the transmission of the objective combination severely. Furthermore the objectives are not rated for wavelengths below ?400 nm? raising the question of if their achromacity still is a viable assumption.\newline
The objectives are replaced with  uncoated 125 nm plano-convex UV fused silica (UVFS) lenses from EKSMA\footnote{?110-1216E?}, with the convex side placed such that the collimated beam enters or exits the convex side. This means that the planar sides of the lenses point towards the sample.\newline
The plano-convex lenses are not achromatic and their chromatic behaviour is described by the thin lens approximation.
\subsection{SHG stage}
\subsubsection{Summary}
To achieve a stable second harmonic generation (SHG) a telescope was built with two spherical lenses. The front lens is a 85 mm lens in a mount adjustable for horizontal and vertical displacement orthogonal to the beam path and the rear lens is a 75 mm UVFS lens on a linear stage, to provide a degree of freedom for the focus of the probe beam, which is necessary to account for the achromacity of the extended setup.
\subsubsection{Operation}
\paragraph{Setting up}
The task of setting up the SHG stage is non trivial regarding proper alignment and linear stage positioning, such that the ~5 cm of movement are sufficient to adjust for the chromatic aberration within the targeted wavelength range for probe. This task often ends up being trial and error in correspondence to checking multiple wavelengths. The cage lenses have to be in their correct positions first.\newline
Irises etc are beneficial if there is a working setup.\newline
Use irises to shape the probe beam to an acceptable Gaussian shape while avoiding diffraction effects. Optimally the incoming beam is level with the table, since any inclination will introduce an astigmatism? depending on the rear lens position.\newline
Begin with the first lens in the path leaving enough space for the linear stage and the SHG crystal to the next static optic. Set lens centered regarding beam. Use the irises in front of the lens to adjust the reflection on the lens surface to return the exact incoming beampath.\newline
Place the rear stage lens and once again adjust for center position. One may again use the reflection and additionally the transmitted beam to tell if the stage is parallel to the beam path. The center of the beam should not move when translating the stage and the expansion of the beam diameter should be as symmetric as possible. Adjusting via the reflection is not such a clear indication of alignment here, as there are multiple convex surfaces in play at this point, leading to multiple focal beams, what may not be in full accordance with each other. For this usually it was chosen to get the highest intensity spot, which refers to the final glass/air surface of the second lens, in the incoming beam path. This spot is also focused, which makes adjustment easier.\newline
Finally one can adjust mirrors after the stage to get an image on the camera again, as even relatively minute changes in alignment will lead to some beam offset and change in direction. Translating the stage then will give an indication of how well the SHG stage is set up in regards to astigmatism aka non symmetric expansion. However beware that the mirror outcoupling in the cage will also have some influence.
\paragraph{For measurements}
To find SHG intensity approach from the incoming beam direction and adjust SHG crystal until sufficient output power is achieved.\newline
Adjust the linear stage such that there is no lateral movement. The stage slide is mounted in a way that can rotate a bit if a rotation force is applied. It may be necessary to do this and have the stage "snap back" to make sure it will not move during the measurement. This can be checked after the measurement by checking the overlap again.
\subsection{Detector Setup}
The detectors themselves are the same as in ...cite robert.... For the measurements detector A was used at all times in combination with a picoscope ???. An unspecified prism UV translucent, equal sided prism is also inserted to remove any remainders of the fundamental probe wavelength at the detector. To improve this separation an extra iris may be used immediately in front of the detector.
\section{Software}
\subsection{Overlap Compensation}
Opposed to the old setup, where the achromatic objectives guaranteed relatively consistent overlap of pump and probe, the UV setup has the need to correct the variation between the single measurements. \newline
Relevant for this are the beam size of the probe as well as the overlap of the pump and the probe beam in the sample. Variations in those parameters are induced by change of the beamshape due to SHG generation, imperfect alignmenent of the SHG telescope and correction of the chromatic abberation for different wavelength settings of the probe.\newline


\subsection{Artray-LiveFitting}
As the Artray $\mathrm{ARTCAM-092UV-WOM}$ could not be interfaced with matlab using the webcam library a C\# program was created based on the manufacturers example program.
Issues with our system are that for our specific camera or system setup, it is not decided which part has the problem, a periodic variation of low intensity signal is reported by the C\# based program, while this periodic variation is missing in the official software. Attempts were made to troubleshoot this in correspondence with the manufacturer, who were very forthcoming in the matter, but it could not be solved so far. To overcome this problem a simple background subtraction has been devised as sufficient. The background has been recorded with the lens cap on and....
\newline
For all measurements version 1.2 were used, which uses ALGLIB (version number)\footnote{ALGLIB for C\# licensed as per GPL 2} for fitting a Gaussian on for the beamshape.

\section{Examination of reliability}
Some tests have been done to ascertain the validity and stability of the corrections devised for the UV extended TAM.

\end{document}